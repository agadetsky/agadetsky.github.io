% !TEX TS-program = xelatex
% !TEX encoding = UTF-8 Unicode
% -*- coding: UTF-8; -*-
% vim: set fenc=utf-8

%%%%%%%%%%%%%%%%%%%%%%%%%%%%%%%%%%%%%%%%%%%%%%%%%%%%%%%%%%%%%%%%%
%% SIMPLE-RESUME-CV
%% <https://github.com/zachscrivena/simple-resume-cv>
%% This is free and unencumbered software released into the
%% public domain; see <http://unlicense.org> for details.
%%%%%%%%%%%%%%%%%%%%%%%%%%%%%%%%%%%%%%%%%%%%%%%%%%%%%%%%%%%%%%%%%

% See "README.md" for instructions on compiling this document.

\documentclass[letterpaper,MMMyyyy,nonstopmode]{simpleresumecv}
% Class options:
% a4paper, letterpaper, nonstopmode, draftmode
% MMMyyyy, ddMMMyyyy, MMMMyyyy, ddMMMMyyyy, yyyyMMdd, yyyyMM, yyyy

%%%%%%%%%%%%%%%%%%%%%%%%%%%%%%%%%%%%%%%%%%%%%%%%%%%%%%%%%%%%%%%%%
%% PREAMBLE.
%%%%%%%%%%%%%%%%%%%%%%%%%%%%%%%%%%%%%%%%%%%%%%%%%%%%%%%%%%%%%%%%%

% CV Info (to be customized).
\newcommand{\CVAuthor}{Artyom Gadetsky}

% PDF settings and properties.
\hypersetup{
pdfauthor={\CVAuthor},
pdfcreator={XeLaTeX},
% pdfproducer={},
pdfkeywords={},
unicode=true,
% bookmarks=true,
bookmarksopen=true,
pdfstartview=FitH,
pdfpagelayout=OneColumn,
pdfpagemode=UseOutlines,
hidelinks,
breaklinks}

% Shorthand.
\newcommand{\Code}[1]{\mbox{\textbf{#1}}}
\newcommand{\CodeCommand}[1]{\mbox{\textbf{\textbackslash{#1}}}}

%%%%%%%%%%%%%%%%%%%%%%%%%%%%%%%%%%%%%%%%%%%%%%%%%%%%%%%%%%%%%%%%%
%% ACTUAL DOCUMENT.
%%%%%%%%%%%%%%%%%%%%%%%%%%%%%%%%%%%%%%%%%%%%%%%%%%%%%%%%%%%%%%%%%

\begin{document}

%%%%%%%%%%%%%%%
% TITLE BLOCK %
%%%%%%%%%%%%%%%

\Title{\CVAuthor}

\begin{SubTitle}
PhD student at \href{https://brbiclab.epfl.ch}{MLBio}
\href{https://www.epfl.ch/schools/ic/}{EPFL}, Lausanne, Switzerland
\par
\href{mailto:artygadetsky@yandex.ru}{artygadetsky@yandex.ru}
\,\SubBulletSymbol\,
\href{https://github.com/agadetsky}{https://github.com/agadetsky}
\,\SubBulletSymbol\,\href{https://agadetsky.github.io}{https://agadetsky.github.io}\,\SubBulletSymbol\,
+7\,(925)\,326-70-02,\ +41\,(0)\,76-270-64-13,
\end{SubTitle}

\begin{Body}

%%%%%%%%%%%%%%%
%% EDUCATION %%
%%%%%%%%%%%%%%%

\Section
{Education}
{Education}
{PDF:Education}

\Entry
\href{https://www.epfl.ch/}
{\textbf{Swiss Federal Institute of Technology (EPFL)}}
\hfill
\DatestampYM{2022}{09} --
\DatestampYM{2026}{12} (expected)
\begin{Detail}
Ph.D. in
\href{https://www.epfl.ch/schools/ic/}
{Computer and Communication Sciences};\\Scientific advisor: \href{http://brbiclab.epfl.ch}{Prof. Maria Brbić}
\end{Detail}
\Gap

\Entry
\href{https://www.hse.ru/en/}
{\textbf{National Research University Higher School of Economics}}
\hfill
\DatestampYM{2018}{09} --
\DatestampYM{2020}{08}
\par
\href{https://www.skoltech.ru/en/}{\textbf{Skolkovo Institute of Science and Technology}}
\par
\begin{Detail}
Double M.S. with honors in
\href{https://www.hse.ru/en/ma/sltheory/}
{Math of Machine Learning};\\
Scientific advisor: \href{http://bayesgroup.org}{Prof. Dmitry Vetrov}
\end{Detail}
\Gap

\Entry
\href{https://www.hse.ru/en/}
{\textbf{National Research University Higher School of Economics}}
\hfill
\DatestampYM{2014}{09} --
\DatestampYM{2018}{08}
\begin{Detail}
B.S. with honors in \href{https://www.hse.ru/en/ba/ami/}{Computer Science, Machine Learning Major};\\
Scientific advisor: \href{http://bayesgroup.org}{Prof. Dmitry Vetrov}
\end{Detail}

%%%%%%%%%%%%%%%%%%%%%%%%%%
%% Research Internships %%
%%%%%%%%%%%%%%%%%%%%%%%%%%

\Section
{Research Internships}
{Research Internships}
{PDF:ResearchInternships}

\Entry
\href{http://apple.com}
{\textbf{Apple @ Zurich, Switzerland}}
\hfill \DatestampYM{2025}{06} --
\DatestampYM{2025}{09}
\begin{Detail}
RL post-training and agentic capabilities @ Apple Foundation Models team
\end{Detail}
\Gap

%\BigGap

%%%%%%%%%%%%%%%%%%
%% PUBLICATIONS %%
%%%%%%%%%%%%%%%%%%

\Section
{Publications}
{Publications}
{PDF:Publications}

\Entry
\hfill
(\textbf{*} denotes equal contribution)

\SubSection
{Conferences}
{Conferences}
{PDF:Conferences}
\Gap

% Declare a new group to limit the scope of \MaxNumberedItem to this subsection.
\Entry
\href{https://openreview.net/forum?id=ohJxgRLlLt}{\textit{Large (Vision) Language Models are Unsupervised In-Context Learners}}
\hfill ICLR 2025
\begin{Detail}
\textbf{Artyom Gadetsky*}, Andrei Atanov*, Yulun Jiang*,\\ Zhitong Gao, Ghazal Hosseini Mighan, Amir Zamir, Maria Brbić
\end{Detail}
\Gap

\Entry
\href{https://proceedings.mlr.press/v235/gadetsky24a.html}{\textit{Let Go of Your Labels with Unsupervised Transfer}} \hfill ICML 2024
\begin{Detail}
\textbf{Artyom Gadetsky*}, Yulun Jiang*, Maria Brbić
\end{Detail}
\Gap

\Entry
\href{https://proceedings.mlr.press/v235/grcic24a.html}{\textit{Fine-grained Classes and How to Find Them}} \hfill ICML 2024
\begin{Detail}
Matej Grcić*, \textbf{Artyom Gadetsky*}, Maria Brbić
\end{Detail}
\Gap

\Entry
\href{https://proceedings.neurips.cc/paper_files/paper/2023/hash/be38c74290c251820e396680a82ce12d-Abstract-Conference.html}{\textit{The Pursuit of Human Labeling: A New Perspective on Unsupervised Learning}} \hfill NeurIPS 2023
\begin{Detail}
\textbf{Artyom Gadetsky}, Maria Brbić \hfill(spotlight)
\end{Detail}
\Gap

\Entry
\href{https://proceedings.neurips.cc/paper/2021/hash/5b658d2a925565f0755e035597f8d22f-Abstract.html}{\textit{Leveraging Recursive Gumbel-Max Trick for Approximate Inference}} \hfill NeurIPS 2021 \\ \href{https://proceedings.neurips.cc/paper/2021/hash/5b658d2a925565f0755e035597f8d22f-Abstract.html}{\textit{in Combinatorial Spaces}}
\begin{Detail}
Kirill Struminsky*, \textbf{Artyom Gadetsky*}, Denis Rakitin*, \\ Danil Karpushkin, Dmitry Vetrov
\end{Detail}
\Gap

\Entry
\href{https://ojs.aaai.org/index.php/AAAI/article/view/6572}{\textit{Low-variance Black-box Gradient Estimates for the Plackett-Luce Distribution}} \hfill AAAI 2020
\begin{Detail}
\textbf{Artyom Gadetsky*}, Kirill Struminsky*, \hfill(oral) \\ Chris Robinson, Novi Quadrianto, Dmitry Vetrov
\end{Detail}
\Gap

\Entry
\href{https://aclanthology.org/P18-2043/}{\textit{Conditional Generators of Words Definitions}} \hfill ACL 2018
\begin{Detail}
\textbf{Artyom Gadetsky}, Ilya Yakubovsky, Dmitry Vetrov
\end{Detail}
\BigGap

\SubSection
{Workshops}
{Workshops}
{PDF:Workshops}
\Gap

\Entry
\href{http://bayesiandeeplearning.org/2019/}{\textit{Low-variance Gradient Estimates for the Plackett-Luce Distribution}} \hfill NeurIPS 2019 BDL
\begin{Detail}
\textbf{Artyom Gadetsky*}, Kirill Struminsky*, \hfill (spotlight) \\ Chris Robinson, Novi Quadrianto, Dmitry Vetrov
\end{Detail}
%\Item
%In proceedings of the Bayesian Deep Learning NeurIPS 2019 Workshop
%%%%%%%%%%%%%%%%%%%%%%%%%%%%%%%%%%%%%%%%%%%%
%% AWARDS %%
%%%%%%%%%%%%%%%%%%%%%%%%%%%%%%%%%%%%%%%%%%%%

\Section
{Awards}
{Awards}
{PDF:Awards}

\Entry
\href{https://www.epfl.ch/education/phd/edic-computer-and-communication-sciences/edic-for-phd-students/}{EPFL EDIC Fellowship 2022-2023}
\begin{Detail}
Fellowship for highly selected first year Ph.D. students
\end{Detail}
\Gap

\Entry
\href{https://yandex.com/scholarships/}{Yandex ML Prize 2020}
\begin{Detail}
Award for outstanding young researchers from CIS region (highly competitive)
\end{Detail}
\Gap

\Entry
\href{https://cs.hse.ru/en/stipend/}{The Ilya Segalovich Scholarship 2019, 2021}
\begin{Detail}
HSE Computer Science Faculty scholarship for outstanding achievements in study and research
\end{Detail}
\Gap

\Entry
\href{https://www.hse.ru/en/scholarships/academic_raised_demo}{Increased State Academic Scholarship 2018-2020}
\begin{Detail}
Scholar in the scientific research section
\end{Detail}
\Gap

\Entry
\href{https://nirs.hse.ru/nirs/}{Open HSE Student Research Paper Competition 2018}
\begin{Detail}
Winner (3rd absolute place)
\end{Detail}
\Gap

\Entry
Best student work at DIALOGUE 2018 conference
\begin{Detail}
For the paper "Conditional Generators of Words Definitions"
\end{Detail}

\newpage

%%%%%%%%%%%%%%%%%%%%%%%%%%%%%%%%%%%%%%%%%%%%
%% Reviewing %%
%%%%%%%%%%%%%%%%%%%%%%%%%%%%%%%%%%%%%%%%%%%%

\Section
{Community Service}
{Community Service}
{PDF:CommunityService}

\Entry
Pre-filtering team at \href{https://ellis.eu}{ELLIS PhD}, \href{https://www.epfl.ch/education/phd/edic-computer-and-communication-sciences/}{EPFL EDIC PhD}
\Gap

\Entry
Reviewer at \href{https://icml.cc}{ICML 2024, 2025}, \href{https://iclr.cc}{ICLR 2024, 2025}, \\
\href{https://neurips.cc}{NeurIPS 2023, 2024}, \href{https://aaai.org/Conferences/AAAI-21/}{AAAI 2021}, \href{https://acl2020.org}{ACL 2020}

%%%%%%%%%%%%%%%%%%%%%%%%%%%%%%%%%%%%%%%%%%%%
%% STUDENT CO-SUPERVISION %%
%%%%%%%%%%%%%%%%%%%%%%%%%%%%%%%%%%%%%%%%%%%%

\Section
{Student Co-supervision}
{Student Co-supervision}
{PDF:Student Co-supervision}

\Entry
\href{https://yljblues.github.io}{Yulun Jiang} (now PhD at EPFL)
\begin{Detail}
%\SubItem
Topic: unsupervised transfer learning
\end{Detail}
\Gap

\Entry
\href{https://johnny1188.github.io}{Jan Sobotka}
\begin{Detail}
%\SubItem
Topic: improving visual language models on the semantic segmentation task
\end{Detail}

\Entry
\href{https://csguoh.github.io}{Hang Guo}
\begin{Detail}
%\SubItem
Topic: training process reward models for LLM test-time scaling without human supervision
\end{Detail}

%%%%%%%%%%%%%%%%%%%%%%%%%%%%%%
%% Industrial Collaboration %%
%%%%%%%%%%%%%%%%%%%%%%%%%%%%%%

\Section
{Industrial Collaboration}
{Industrial Collaboration}
{PDF:IndustrialCollaboration}

\Entry
\textbf{ZEISS Group}, Munich, Germany
\hfill
\DatestampY{2024} --
\DatestampY{2025}
\begin{Detail}
Conducting research on unsupervised adaptation techniques for visual language foundation \\
models on image segmentation; advisor  \href{https://brbiclab.epfl.ch/team/}{Maria Brbic}; external manager \href{https://www.linkedin.com/in/jascha-tempeler-b7b992191/}{Jascha Tempeler}
\end{Detail}
\Gap

\Entry
\textbf{Huawei Research}, Moscow, Russia
\hfill
\DatestampY{2020} --
\DatestampY{2022}
\begin{Detail}
Conducting research on using synthetic data for knowledge \\
distillation under domain gap; advisor: \href{https://bayesgroup.ru/people/dmitry-vetrov/}{Dmitry Vetrov}
\end{Detail}
\Gap

\Entry
\textbf{Sberbank-HSE Laboratory}, Moscow, Russia
\hfill
\DatestampY{2017} --
\DatestampY{2020}
\begin{Detail}
Conducting research on different text generative modeling tasks.\\
Results were published at ACL 2018. manager: \href{https://www.hse.ru/en/staff/esokolov}{Evgeny Sokolov}.
\end{Detail}

%%%%%%%%%%%%%%%%%%%%%%%%%%%%%%%%%%%%%%%%%%%%
%% TEACHING %%
%%%%%%%%%%%%%%%%%%%%%%%%%%%%%%%%%%%%%%%%%%%%

\Section
{Teaching Experience}
{Teaching Experience}
{PDF:TeachingExperince}

\Entry
Applied Data Analysis at \href{https://edu.epfl.ch/coursebook/en/applied-data-analysis-CS-401}{EPFL} (largest course on campus!)
\hfill
Fall 2024
\Gap

\Entry
Calculus II at \href{https://edu.epfl.ch/coursebook/en/analysis-ii-MATH-106-E}{EPFL}
\hfill
Spring 2024
\Gap

\Entry
Deep Learning in Biomedicine at \href{https://edu.epfl.ch/coursebook/en/deep-learning-in-biomedicine-CS-502}{EPFL}
\hfill
Fall 2023
\Gap

\Entry
Transfer Learning and Meta-Learning at \href{https://edu.epfl.ch/coursebook/en/transfer-learning-and-meta-learning-CS-625}{EPFL}
\hfill
Spring 2023
\Gap

\Entry
Continuous Optimization course at \href{https://cs.hse.ru/en/}{HSE} and \href{https://cs.msu.ru/en}{MSU}
\hfill
\DatestampY{2018} --
\DatestampY{2022}
\Gap

\Entry
Neurobayesian Methods at \href{https://cs.hse.ru/en/}{HSE}, \href{https://cs.msu.ru/en}{MSU} and \href{https://yandexdataschool.com}{YSDA}
\hfill
\DatestampY{2018} --
\DatestampY{2022}
\Gap

\Entry
Introduction to Deep Learning at \href{https://cs.hse.ru/en/}{HSE}
\hfill
2018-2021
\Gap

\Entry
Data Analysis course at \href{https://cs.hse.ru/en/}{HSE}
\hfill
\DatestampY{2018}
\Gap

\Entry
\href{http://deepbayes.ru}{DeepBayes Summer School}
\hfill
\DatestampY{2017}, \DatestampY{2018}, \DatestampY{2019}
\Gap

\Entry
\href{https://www.coursera.org/learn/bayesian-methods-in-machine-learning}
{Coursera ”Bayesian Methods in Machine Learning” course}
\hfill
2017-2018
\Gap

\Entry
”Bayesian Methods in Machine Learning” course at \href{https://yandexdataschool.com}{YSDA}
and \href{https://cs.msu.ru/en}{CMC MSU}
\hfill
2017-2018
\Gap

\Entry
Machine Learning course at \href{https://cs.hse.ru/en/}{CS HSE}
\hfill
\DatestampYMD{2017}{09}{01} --
\DatestampYMD{2018}{07}{01}
\Gap

%%%%%%%%%%%%%%%%%%
%% OTHER EXPERIENCE %%
%%%%%%%%%%%%%%%%%%

\Section
{Other experience}
{Other experience}
{PDF:OtherExperience}

\Entry
\href{https://www.microsoft.com/en-us/research/event/ai-summer-school-2018/}{Microsoft Research AI Summer School 2018}
\hfill
\DatestampYM{2018}{07}
\Gap

\Entry
Visiting scholar at the University of Sussex, UK
\hfill
\DatestampYM{2019}{07} --
\DatestampYM{2019}{08}, \DatestampYM{2019}{11}
\begin{Detail}
%\Item
Hosted by \href{http://www.sussex.ac.uk/profiles/335583}{Novi Quadrianto} \hfil
\end{Detail}

%%%%%%%%%%%%%%%%%%%%%%%%%%%%%%%%%%%%%%%%%%%%
%% Skills %%
%%%%%%%%%%%%%%%%%%%%%%%%%%%%%%%%%%%%%%%%%%%%

\Section
{Skills}
{Skills}
{PDF:Skills}

\Entry Research, Machine Learning, Deep Learning, Statistics, Algorithms \& Data Structures, Python, PyTorch, TensorFlow Basics, C++ Basics


%%%%%%%%%%%%%%%%%%%%%%%%
%% RESEARCH INTERESTS %%
%%%%%%%%%%%%%%%%%%%%%%%%

\Section
{Research Interests}
{Research Interests}
{PDF:ResearchInterests}

\Entry LLM Agents, Reinforcement Learning, Reasoning, Unsupervised Learning, Representation Learning, Foundation Models, Generative AI, Probabilistic Modeling, Stochastic Gradient Estimation, Structured Discrete Variables, Monte Carlo Methods, Deep Learning, Machine Learning.


\iffalse
%%%%%%%%%%%%%%%
%% LANGUAGES %%
%%%%%%%%%%%%%%%

\Section
{Languages}
{Languages}
{PDF:Languages}

\BulletItem
Russian: Mother tongue.

\Gap
\BulletItem
English: Fluent

%%%%%%%%%%%%
%% SKILLS %%
%%%%%%%%%%%%

\Section
{Skills}
{Skills}
{PDF:Skills}

\BulletItem
Machine Learning, Deep Learning, Probabilistic Machine Learning

\BulletItem
Python, Pytorch, Feel free with Linux
\fi

\iffalse
%%%%%%%%%%%%%%%%%%%%%%%%%%%%%%%%%%%%%%%%
%% THIS IS A SECTION WITH USAGE NOTES %%
%%%%%%%%%%%%%%%%%%%%%%%%%%%%%%%%%%%%%%%%

% Declare a new group to limit the scope of \color to this section.
\begingroup
\color{red}

\Section
{This is a\newline
Section\newline
With\newline
Usage Notes}
{This is a Section With Usage Notes (For PDF Bookmark)}
{PDF:ThisIsASectionWithUsageNotes:ForPDFLink}

\SubSection
{This is a SubSection}
{This is a SubSection (For PDF Bookmark)}
{PDF:ThisIsASubSection:ForPDFLink}

\Gap
\BulletItem
Use \CodeCommand{Section\{a\}\{b\}\{c\}} and
\CodeCommand{SubSection\{a\}\{b\}\{c\}}
to create sections and subsections, where
\Code{a} is the heading displayed on the page,
\Code{b} is the PDF bookmark heading, and
\Code{c} is the internal PDF link (must be unique).
Sections and subsections will appear in the PDF bookmarks.
Note the CamelCase command names.

\Gap
\BulletItem
Use
\CodeCommand{Entry},
\CodeCommand{BulletItem},
\CodeCommand{SubBulletItem},
\CodeCommand{Item},
\CodeCommand{SubItem},
\CodeCommand{NumberedItem},
etc.,
to create entries in the main body of the CV.

\Gap
\BulletItem
Enclose entry details between
\CodeCommand{begin\{Detail\}} and
\CodeCommand{end\{Detail\}}
so that they are typeset in a smaller font.
\begin{Detail}
\Item
This is an example of entry detail text enclosed in a \Code{Detail} environment.
\end{Detail}

\Gap
\BulletItem
Use \CodeCommand{Gap} and \CodeCommand{BigGap} to insert vertical spaces between entries to improve layout.

\BigGap
\SubSection
{This is Another SubSection}
{This is Another Subsection (For PDF Bookmark)}
{PDF:ThisIsAnotherSubSection:ForPDFLink}

\Gap
\Entry
This is a plain \CodeCommand{Entry},
followed by an \CodeCommand{hfill} and a date range
\hfill
\DatestampYM{2015}{10} --
\DatestampYM{2015}{12}

\Gap
\BulletItem
This is a \CodeCommand{BulletItem}.
\Item
This is an \CodeCommand{Item}, which has no bullet.
Note the alignment with the \CodeCommand{BulletItem} above.

\Gap
\SubBulletItem
This is a \CodeCommand{SubBulletItem}.
\SubItem
This is a \CodeCommand{SubItem}, which has no bullet.
Note the alignment with the \CodeCommand{SubBulletItem} above.

\Gap
\NumberedItem{[42]}
This is a \CodeCommand{NumberedItem}.
Change the value of the macro \CodeCommand{MaxNumberedItem} to adjust the indentation width.

\BigGap
\SubSection
{Line, Paragraph, and Page Breaks}
{Line, Paragraph, and Page Breaks (For PDF Bookmark)}
{PDF:LineParagraphAndPageBreaks:ForPDFLink}

\Gap
\BulletItem
To create a new line within the same paragraph (i.e., preserving the same paragraph indentation), use \CodeCommand{newline} instead of \CodeCommand{\textbackslash};
the latter will reset the paragraph indentation.

\Gap
\BulletItem
To create a new paragraph, use \CodeCommand{par} or simply leave an empty line.
Paragraph indentations (from
\CodeCommand{Entry},
\CodeCommand{BulletItem},
\CodeCommand{SubBulletItem},
\CodeCommand{Item},
\CodeCommand{SubItem},
\CodeCommand{NumberedItem},
etc.) do not carry across different paragraphs.

\Gap
\BulletItem
To create a new page, use \CodeCommand{newpage}.

\BigGap
\SubSection
{Dates}
{Dates (For PDF Bookmark)}
{PDF:Dates:ForPDFLink}

\Gap
\BulletItem
Use the following macros to specify and display dates consistently:
\SubBulletItem
\CodeCommand{DatestampYMD\{yyyy\}\{MM\}\{dd\}}
(e.g., \CodeCommand{DatestampYMD\{2008\}\{01\}\{15\}})
\SubBulletItem
\CodeCommand{DatestampYM\{yyyy\}\{MM\}}
(e.g., \CodeCommand{DatestampYM\{2008\}\{01\}})
\SubBulletItem
\CodeCommand{DatestampY\{yyyy\}}
(e.g., \CodeCommand{DatestampY\{2008\}})

\Gap
\BulletItem
Change the date format option passed to the document class to adjust how dates are displayed throughout the document:
\SubBulletItem
\Code{MMMyyyy} (``Jan~2008'')
\SubBulletItem
\Code{ddMMMyyyy} (``15~Jan~2008'')
\SubBulletItem
\Code{MMMMyyyy} (``January~2008'')
\SubBulletItem
\Code{ddMMMMyyyy} (``15~January~2008'')
\SubBulletItem
\Code{yyyyMMdd} (``2008-01-15'')
\SubBulletItem
\Code{yyyyMM} (``2008-01'')
\SubBulletItem
\Code{yyyy} (``2008'')

\endgroup

\fi

\end{Body}

\end{document}
